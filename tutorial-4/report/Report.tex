%%%%%%%%%%%%%%%%%%%%%%%%%%%%%%%%%%%%%%%%%%%%%%%%%%%%%%%%%%%%%
%% HEADER
%%%%%%%%%%%%%%%%%%%%%%%%%%%%%%%%%%%%%%%%%%%%%%%%%%%%%%%%%%%%%
\pdfminorversion=7
\pdfcompresslevel=1 % schnellste Kompression
%\pdfcompresslevel=9 % beste Kompression
\pdfimageresolution=600

\documentclass[12pt,bibliography=totoc]{scrreprt}

\usepackage{i17report}

\usepackage{scrhack} % used since scrreprt produces a warning when using totoc

%% Language %%%%%%%%%%%%%%%%%%%%%%%%%%%%%%%%%%%%%%%%%%%%%%%%%
\usepackage[USenglish]{babel}     %%%% english
%\usepackage[ngerman]{babel}      %%%% german
\usepackage[T1]{fontenc}
\usepackage[ansinew]{inputenc}

\usepackage{lmodern} %Type1-font for non-english texts and characters

%% Hyperrefs and URLs %%%%%%%%%%%%%%%%%%%%%%%%%%%%%%%%%%%%%%%%%%%%%%%%%
\usepackage[pdftex,                %%% hyper-references for pdftex
bookmarks=true,%                   %%% generate bookmarks \ldots
bookmarksnumbered=true,%           %%% \ldots with numbers
hypertexnames=false,%              %%% needed for correct links to figures !!!
hyperfootnotes=false,
breaklinks=true,%                  %%% breaks lines, but links are very small
linkcolor=red,%
linkbordercolor={1 0 1}%
]{hyperref}  %%% blue frames around links

%% Packages for Graphics & Figures %%%%%%%%%%%%%%%%%%%%%%%%%%
\usepackage{graphicx}
\usepackage[export]{adjustbox}
\usepackage[most]{tcolorbox}
\usepackage{float}  % for [H]

%% Packages for Tables %%%%%%%%%%%%%%%%%%%%%%%%%%%%%%%%%%%%%
\usepackage{multirow}
\usepackage{ltablex}
\usepackage{booktabs}
\newcolumntype{C}[1]{>{\centering\arraybackslash}p{#1}}
\usepackage{tabularx}

%% Packages for Coloring %%%%%%%%%%%%%%%%%%%%%%%%%%%%%%%%%%%
\usepackage{color, xcolor}
\definecolor{material_blue}{HTML}{027BE3}
\definecolor{material_red}{HTML}{F44336}
\definecolor{material_green}{HTML}{4CAF50}
\definecolor{material_orange}{HTML}{FF9800}

%% Packages for Math %%%%%%%%%%%%%%%%%%%%%%%%%%%%%%%%%%%%%%%%
\usepackage{amsmath}
\usepackage{amssymb}
\usepackage{amsthm}
\usepackage{amsfonts}
\usepackage{mathtools}
\usepackage{units}
\usepackage{relsize,exscale}
\usepackage{parskip}

%% Configuration for JavaScript %%%%%%%%%%%%%%%%%%%%%%%%%%%%%%%%%%%%%%%%
\lstset{
    language=[LaTeX]TeX,  % generic, no syntax highlighting errors
    basicstyle=\ttfamily\small,
    breaklines=true,
    frame=single
}

%% Defined commands %%%%%%%%%%%%%%%%%%%%%%%%%%%%%%%%%%%%%%%%%
\newcommand{\bigint} [3] {\mathop{\mathlarger{\int\limits_{#1}^{#2}}} {#3}}  % a big integral

%%%%%%%%%%%%%%%%%%%%%%%%%%%%%%%%%%%%%%%%%%%%%%%%%%%%%%%%%%%%%
%% DOCUMENT
%%%%%%%%%%%%%%%%%%%%%%%%%%%%%%%%%%%%%%%%%%%%%%%%%%%%%%%%%%%%%
\begin{document}

\pagestyle{empty} %No headings for the first pages.

%% Title Page %%%%%%%%%%%%%%%%%%%%%%%%%%%%%%%%%%%%%%%%%%%%%%%
\title{Tele-Experiment}
\subject{Kinematics and Navigation of the MERLIN Robot}

\author{Kaung Sett Thu}
\matrnr{12503548}
\submission{\today}

\maketitle

%% Inhaltsverzeichnis %%%%%%%%%%%%%%%%%%%%%%%%%%%%%%%%%%%%%%%
\tableofcontents
\cleardoublepage %The first chapter should start on an odd page.

\pagestyle{plain}

%% Chapters %%%%%%%%%%%%%%%%%%%%%%%%%%%%%%%%%%%%%%%%%%%%%%%%%
\chapter{The Imaginary Wheel}

The relationship between the steering angle of the imaginary wheel $\phi$, the wheelbase $L$, and the radius of the robot $R$ can be described as follow.
%
\begin{align*}
    \tan \phi &= \dfrac{L}{R}
\end{align*}

\section{Imaginary Wheel $\phi$ in Relation to Outer Wheel $\phi_o$}

The turning radius of the outer wheel from \texttt{ICR} is $R + \frac{D}{2}$. Therefore
%
\begin{align*}
    \tan \phi_o &= \dfrac{L}{R + \dfrac{D}{2}} \\[1em]
    R + \dfrac{D}{2} &= \dfrac{L}{\tan \phi_o} \\[1em]
    R &= \dfrac{L}{\tan \phi_o} - \dfrac{D}{2}
\end{align*}
%
\emph{Substituting formula for R in terms of steering angle of the outer wheel to formula for the steering angle of the imaginary wheel}
%
\begin{align*}
    \tan \phi &= \dfrac{L}{\dfrac{L}{\tan \phi_o} - \dfrac{D}{2}} \\[1em]
    \tan \phi &= \dfrac{1}{\dfrac{1}{\tan \phi_o} - \dfrac{D}{2L}} \\[1em]
    \tan \phi &= \dfrac{\tan \phi_o}{1 - \dfrac{D}{2L} \cdot \tan \phi_o} \\[1em]
    \phi &= \arctan\left(\dfrac{\tan \phi_o}{1 - \dfrac{D}{2L} \cdot \tan \phi_o}\right)
\end{align*}

\section{Imaginary Wheel $\phi$ in Relation to Inner Wheel $\phi_i$}

The turning radius of the inner wheel from \texttt{ICR} is $R - \frac{D}{2}$. Therefore
%
\begin{align*}
    \tan \phi_i &= \dfrac{L}{R - \dfrac{D}{2}} \\[1em]
    R - \dfrac{D}{2} &= \dfrac{L}{\tan \phi_i} \\[1em]
    R &= \dfrac{L}{\tan \phi_i} + \dfrac{D}{2}
\end{align*}
%
\emph{Substituting formula for R in terms of steering angle of the inner wheel to formula for the steering angle of the imaginary wheel}
%
\begin{align*}
    \tan \phi &= \dfrac{L}{\dfrac{L}{\tan \phi_i} + \dfrac{D}{2}} \\[1em]
    \tan \phi &= \dfrac{1}{\dfrac{1}{\tan \phi_i} + \dfrac{D}{2L}} \\[1em]
    \tan \phi &= \dfrac{\tan \phi_i}{1 + \dfrac{D}{2L} \cdot \tan \phi_i} \\[1em]
    \phi &= \arctan\left(\dfrac{\tan \phi_i}{1 + \dfrac{D}{2L} \cdot \tan \phi_i}\right)
\end{align*}

\chapter{Calculating the Robot Pose using the Odometry (Dead Reckoning)}
\section{Pose Estimation in relation to the Velocity $v_{CR}$}

The angular velocity of the robot $\omega(t)$ in relation to the velocity of the point $CR$ $v_{CR}(t)$ can be expressed as
%
\begin{align*}
    \omega(t) &= \dfrac{v_{CR}(t)}{R_{CR}}
\end{align*}

Since $v_{CR}$ is constant, angular velocity $\omega$ can be expressed as a constant.
%
\begin{align*}
    \omega &= \dfrac{v_{CR}}{R_{CR}}
\end{align*}

The relation of the steering angle $\phi$ to the radius from Instantenous Center of Rotation (ICR) $R_{CR}$ can be rearranged to express it in terms of the radius.
%
\begin{align*}
    \tan \phi &= \dfrac{L}{R_{CR}} \\[1em]
    R_{CR} &= \dfrac{L}{\tan \phi}
\end{align*}

Replacing the formula for radius $R_{CR}$ into the formula for angular velocity $\omega$,
%
\begin{align*}
    \omega &= \dfrac{v_{CR}}{\dfrac{L}{\tan \phi}} \\[1em]
    &= \dfrac{v_{CR}}{L} \cdot \tan \phi
\end{align*}

The pose estimation for the orientation of the robot $\theta$ can be expressed as
%
\begin{align*}
    \theta(t) &= \theta_0 + \int_{0}^{t} \omega \, d\tau \\[1em]
    &= \theta_0 + \omega \big[\tau\big]_{0}^{t} \\[1em]
    &= \theta_0 + \omega \cdot t
\end{align*}

The pose estimation for $x$ coordinate of the robot can be expressed as
%
\begin{align*}
    x(t) &= x_0 + \int_{0}^{t} v_{CR} \cdot \cos(\theta(\tau)) \, d\tau \\[1em]
    &= x_0 + v_{CR} \int_{0}^{t} \cos(\theta_0 + \omega \cdot \tau) \, d\tau
\end{align*}

\emph{Integrating $\cos(\theta_0 + \omega \cdot \tau)$ using U-Substitution}

Let
%
\begin{align*}
    u = \theta_0 + \omega \cdot \tau
\end{align*}
%
\begin{align*}
    \dfrac{du}{d\tau} &= \omega \\[1em]
    d\tau &= \dfrac{1}{\omega} \cdot du
\end{align*}

When $\tau = 0$,
%
\begin{align*}
    u = \theta_0
\end{align*}

When $\tau = t$,
%
\begin{align*}
    u = \theta_0 + \omega \cdot t
\end{align*}
%
\begin{align*}
    x(t) &= x_0 + v_{CR} \int_{\theta_0}^{\theta_0 + \omega \cdot t} \cos(u) \, \dfrac{1}{\omega} \cdot du \\[1em]
    &= x_0 + \dfrac{v_{CR}}{\omega} \int_{\theta_0}^{\theta_0 + \omega \cdot t} \cos(u) \, \cdot du \\[1em]
    &= x_0 + \dfrac{v_{CR}}{\omega} \Big[\sin(u)\Big]_{\theta_0}^{\theta_0 + \omega \cdot t} \\[1em]
    &= x_0 + \dfrac{v_{CR}}{\omega} \big(\sin(\theta_0 + \omega \cdot t) - \sin(\theta_0) \big)
\end{align*}

Replacing the formula for the angular velocity $\omega$ into the equation,
%
\begin{align*}
    x(t) &= x_0 + v_{CR} \cdot \dfrac{R_{CR}}{v_{CR}} \big(\sin(\theta_0 + \omega \cdot t) - \sin(\theta_0) \big) \\[1em]
    &= x_0 + R_{CR} \big(\sin(\theta_0 + \omega \cdot t) - \sin(\theta_0) \big)
\end{align*}

The pose estimation for $y$ coordinate of the robot can be expressed as
%
\begin{align*}
    y(t) &= y_0 + \int_{0}^{t} v_{CR} \cdot \sin(\theta(\tau)) \, d\tau \\[1em]
    &= y_0 + v_{CR} \int_{0}^{t} \sin(\theta_0 + \omega \cdot \tau) \, d\tau
\end{align*}

\emph{Integrating $\sin(\theta_0 + \omega \cdot \tau)$ using U-Substitution}

Let
%
\begin{align*}
    u = \theta_0 + \omega \cdot \tau
\end{align*}
%
\begin{align*}
    \dfrac{du}{d\tau} &= \omega \\[1em]
    d\tau &= \dfrac{1}{\omega} \cdot du
\end{align*}

When $\tau = 0$,
%
\begin{align*}
    u = \theta_0
\end{align*}

When $\tau = t$,
%
\begin{align*}
    u = \theta_0 + \omega \cdot t
\end{align*}
%
\begin{align*}
    y(t) &= y_0 + v_{CR} \int_{\theta_0}^{\theta_0 + \omega \cdot t} \sin(u) \, \dfrac{1}{\omega} \cdot du \\[1em]
    &= y_0 + \dfrac{v_{CR}}{\omega} \int_{\theta_0}^{\theta_0 + \omega \cdot t} \sin(u) \, \cdot du \\[1em]
    &= y_0 + \dfrac{v_{CR}}{\omega} \Big[-\cos(u)\Big]_{\theta_0}^{\theta_0 + \omega \cdot t} \\[1em]
    &= y_0 + \dfrac{v_{CR}}{\omega} \big(\cos(\theta_0) - \cos(\theta_0 + \omega \cdot t) \big)
\end{align*}

Replacing the formula for the angular velocity $\omega$ into the equation,
%
\begin{align*}
    y(t) &= y_0 + v_{CR} \cdot \dfrac{R_{CR}}{v_{CR}} \big(\cos(\theta_0) - \cos(\theta_0 + \omega \cdot t) \big) \\[1em]
    &= y_0 + R_{CR} \big(\cos(\theta_0) - \cos(\theta_0 + \omega \cdot t) \big)
\end{align*}

The pose estimation in relation to the velocity $v_{CR}$ can collectively be expressed in a matrix form.
%
\begin{align*}
    p(t) =
    \begin{bmatrix}
        x(t) \\
        y(t) \\
        \theta(t)
    \end{bmatrix} =
    \begin{bmatrix}
        x_0 \\
        y_0 \\
        \theta_0
    \end{bmatrix}
    +
    \begin{bmatrix}
        R_{CR} \big(\sin(\theta_0 + \omega t) - \sin\theta_0 \big) \\
        R_{CR} \big(\cos\theta_0 - \cos(\theta_0 + \omega t) \big) \\
        \omega \cdot t
    \end{bmatrix}
\end{align*}

\section{Pose Estimation in relation to Velocity at Imaginary Wheel $v_{IM}$}

In the right-angled triangle formed by the rear axle center, the imaginary front wheel, and the instantaneous center of rotation (ICR), the distance from the ICR to the imaginary wheel, $R_{IM}$ is the hypotenuse of this triangle, while the wheelbase $L$ corresponds to the side opposite the steering angle $\phi$. The steering angle $\phi$ can therefore be expressed by the use of the $sine$ ratio.
%
\begin{align*}
    \sin \phi = \dfrac{L}{R_{IM}} 
\end{align*}

The equation can be rearranged to express it in terms of the radius $R_{IM}$.
%
\begin{align*}
    R_{IM} = \dfrac{L}{\sin \phi}
\end{align*}

An angular velocity $\omega$ can be expressed as follows.
%
\begin{align*}
    \omega &= \dfrac{v_{IM}}{R_{IM}} \\[1em]
    &= \dfrac{v_{IM}}{\dfrac{L}{\sin \phi}} \\[1em]
    &=  \dfrac{v_{IM} \cdot \sin \phi}{L}
\end{align*}

Since the angular velocity $\omega$ remains the same for both the rear axle and the front wheel, the relationship between velocity at the rear axis $v_{CR}$ and the imaginary wheel $v_{IM}$ can be calculated.
%
\begin{align*}
    \dfrac{v_{CR}}{R_{CR}} &= \dfrac{v_{IM}}{R_{IM}} \\[1em]
    v_{CR} &= \dfrac{v_{IM}}{R_{IM}} \cdot R_{CR} \\[1em]
    v_{CR} &= \dfrac{v_{IM} \cdot \sin \phi}{L} \cdot \dfrac{L}{\tan \phi} \\[1em]
    v_{CR} &= v_{IM} \cdot \cos \phi
\end{align*}

Using this relationship between $v_{CR}$ and $v_{IM}$, the pose estimation for $x(t)$ and $y(t)$ can be expressed in terms of $v_{IM}$ as
%
\begin{align*}
    x(t) &= x_0 + \int_{0}^{t} (v_{IM} \cdot \cos \phi) \cdot \cos(\theta(\tau)) \, d\tau \\[1em]
    y(t) &= y_0 + \int_{0}^{t} (v_{IM} \cdot \cos \phi) \cdot \sin(\theta(\tau)) \, d\tau
\end{align*}

Since angular velocity $\omega$ is constant the pose estimation for orientation $\theta(t)$ remains the same.

Following the same integration steps as section 2.1 for $x(t)$ and $y(t)$, we get
%
\begin{align*}
    x(t) &= x_0 + \dfrac{v_{IM} \cdot \cos \phi}{\omega} \big(\sin(\theta_0 + \omega \cdot t) - \sin(\theta_0) \big) \\[1em] 
    &= x_0 + (v_{IM} \cdot \cos \phi) \cdot \dfrac{L}{v_{IM} \cdot \sin \phi} \big(\sin(\theta_0 + \omega \cdot t) - \sin(\theta_0) \big) \\[1em] 
    &= x_0 + \dfrac{L \cdot \cos \phi}{\sin \phi} \big(\sin(\theta_0 + \omega \cdot t) - \sin(\theta_0) \big) \\[1em] 
    &= x_0 + \dfrac{L}{\tan \phi} \big(\sin(\theta_0 + \omega \cdot t) - \sin(\theta_0) \big)
\end{align*}

\begin{align*}
    y(t) &= y_0 + \dfrac{v_{IM} \cdot \cos \phi}{\omega} \big(\cos(\theta_0) - \cos(\theta_0 + \omega \cdot t) \big) \\[1em]
    &= y_0 + (v_{IM} \cdot \cos \phi) \cdot \dfrac{L}{v_{IM} \cdot \sin \phi} \big(\cos(\theta_0) - \cos(\theta_0 + \omega \cdot t) \big) \\[1em]
    &= y_0 + \dfrac{L \cdot \cos \phi}{\sin \phi} \big(\cos(\theta_0) - \cos(\theta_0 + \omega \cdot t) \big) \\[1em]
    &= y_0 + \dfrac{L}{\tan \phi} \big(\cos(\theta_0) - \cos(\theta_0 + \omega \cdot t) \big)
\end{align*}

Since $\dfrac{L}{tan \phi}$ is radius from rear axle $R_{CR}$, we can see that pose estimation in relation to velocity at imaginary wheel $v_{IM}$ has a similar matrix to pose estimation in relation to velocity at rear axle $v_{CR}$
%
\begin{align*}
    p(t) =
    \begin{bmatrix}
        x(t) \\
        y(t) \\
        \theta(t)
    \end{bmatrix} =
    \begin{bmatrix}
        x_0 \\
        y_0 \\
        \theta_0
    \end{bmatrix}
    +
    \begin{bmatrix}
        R_{CR} \big(\sin(\theta_0 + \omega t) - \sin\theta_0 \big) \\
        R_{CR} \big(\cos\theta_0 - \cos(\theta_0 + \omega t) \big) \\
        \omega \cdot t
    \end{bmatrix}
\end{align*}

with angular velocity $\omega$ being expressed in terms of velocity at imaginary wheel $v_{IM}$ as
%
\begin{align*}
    \omega = \dfrac{v_{IM} \cdot \sin \phi}{L}
\end{align*}

\section{Discrete Form of the Robot's Pose in relation to $v_{CR}$}

Based on the matrix derived in section 2.1, assuming constant velocity $v_{CR}$ and constant steering angle $\phi$ over the sampling interval $\Delta t$, the exact discrete-time update of the robot pose can be expressed as
%
\begin{align*}
    p(k+1) =
    \begin{bmatrix}
        x_{k+1} \\
        y_{k+1} \\
        \theta_{k+1}
    \end{bmatrix} =
    \begin{bmatrix}
        x_k \\
        y_k \\
        \theta_k
    \end{bmatrix}
    +
    \begin{bmatrix}
        R_{CR} \big(\sin(\theta_k + \omega \Delta t) - \sin\theta_k \big) \\
        R_{CR} \big(\cos\theta_k - \cos(\theta_k + \omega \Delta t) \big) \\
        \omega \cdot \Delta t
    \end{bmatrix}
\end{align*}

\section{Recurrent Form of Equations}

For recurrent odometry updates under constant steering angle and velocity over the time $\Delta t$, the pose can be updated recurrently by following pipeline of equations

\begin{enumerate}
    \item Update the orientation
        \begin{align*}
            \theta_{k+1} = \theta_k + \dfrac{v_{CR} \cdot \tan \phi}{L} \cdot \Delta t
        \end{align*}
    \item Update the $x$ coordinate
        \begin{align*}
            x_{k+1} = x_k + \dfrac{L}{\tan \phi} \big(\sin \theta_{k+1} - \sin \theta_k \big)
        \end{align*}
    \item Update the $y$ coordinate
        \begin{align*}
            y_{k+1} = y_k + \dfrac{L}{\tan \phi} \big(\cos \theta_k - \cos \theta_{k+1} \big)
        \end{align*} 
\end{enumerate}

In case of a straight line motion where $\phi$ is 0, the equations simplify to

\begin{enumerate}
    \item Update the orientation
        \begin{align*}
            \theta_{k+1} = \theta_k
        \end{align*}
    \item Update the $x$ coordinate
        \begin{align*}
            x_{k+1} = x_k + v_{CR}\cdot \cos \theta_k \cdot \Delta t
        \end{align*}
    \item Update the $y$ coordinate
        \begin{align*}
            y_{k+1} = y_k + v_{CR}\cdot \sin \theta_k \cdot \Delta t
        \end{align*} 
\end{enumerate}

\chapter{Shifting and Rotation}

%%%%%%%%%%%%%%%%%%%%%%%%%%%%%%%%%%%%%%%%%%%%%%%%%%%%%%%%%%%%%
%% BIBLIOGRAPHY AND OTHER LISTS
%%%%%%%%%%%%%%%%%%%%%%%%%%%%%%%%%%%%%%%%%%%%%%%%%%%%%%%%%%%%%
\bibliographystyle{plain}
%\bibliography{Literature}

%%%%%%%%%%%%%%%%%%%%%%%%%%%%%%%%%%%%%%%%%%%%%%%%%%%%%%%%%%%%%
%% Appendix
%%%%%%%%%%%%%%%%%%%%%%%%%%%%%%%%%%%%%%%%%%%%%%%%%%%%%%%%%%%%%
\begin{appendix}

%\input{Appendix}

\end{appendix}

\end{document}

