%%%%%%%%%%%%%%%%%%%%%%%%%%%%%%%%%%%%%%%%%%%%%%%%%%%%%%%%%%%%%
%% HEADER
%%%%%%%%%%%%%%%%%%%%%%%%%%%%%%%%%%%%%%%%%%%%%%%%%%%%%%%%%%%%%
\pdfminorversion=7
\pdfcompresslevel=1 % schnellste Kompression
%\pdfcompresslevel=9 % beste Kompression
\pdfimageresolution=600

\documentclass[12pt,bibliography=totoc]{scrreprt}

\usepackage{i17report}

\usepackage{scrhack} % used since scrreprt produces a warning when using totoc

%% Language %%%%%%%%%%%%%%%%%%%%%%%%%%%%%%%%%%%%%%%%%%%%%%%%%
\usepackage[USenglish]{babel}     %%%% english
%\usepackage[ngerman]{babel}      %%%% german
\usepackage[T1]{fontenc}
\usepackage[ansinew]{inputenc}

\usepackage{lmodern} %Type1-font for non-english texts and characters

%% Hyperrefs and URLs %%%%%%%%%%%%%%%%%%%%%%%%%%%%%%%%%%%%%%%%%%%%%%%%%
\usepackage[pdftex,                %%% hyper-references for pdftex
bookmarks=true,%                   %%% generate bookmarks \ldots
bookmarksnumbered=true,%           %%% \ldots with numbers
hypertexnames=false,%              %%% needed for correct links to figures !!!
hyperfootnotes=false,
breaklinks=true,%                  %%% breaks lines, but links are very small
linkcolor=red,%
linkbordercolor={1 0 1}%
]{hyperref}  %%% blue frames around links

%% Packages for Graphics & Figures %%%%%%%%%%%%%%%%%%%%%%%%%%
\usepackage{graphicx}
\usepackage[export]{adjustbox}
\usepackage[most]{tcolorbox}
\usepackage{float}  % for [H]

%% Packages for Tables %%%%%%%%%%%%%%%%%%%%%%%%%%%%%%%%%%%%%
\usepackage{multirow}
\usepackage{ltablex}
\usepackage{booktabs}
\newcolumntype{C}[1]{>{\centering\arraybackslash}p{#1}}

%% Packages for Coloring %%%%%%%%%%%%%%%%%%%%%%%%%%%%%%%%%%%
\usepackage{color, xcolor}
\definecolor{material_blue}{HTML}{027BE3}
\definecolor{material_red}{HTML}{F44336}
\definecolor{material_green}{HTML}{4CAF50}
\definecolor{material_orange}{HTML}{FF9800}

%% Packages for Math %%%%%%%%%%%%%%%%%%%%%%%%%%%%%%%%%%%%%%%%
\usepackage{amsmath}
\usepackage{amssymb}
\usepackage{amsthm}
\usepackage{amsfonts}
\usepackage{mathtools}
\usepackage{units}
\usepackage{relsize,exscale}
\usepackage{parskip}

%% Defined commands %%%%%%%%%%%%%%%%%%%%%%%%%%%%%%%%%%%%%%%%%
\newcommand{\bigint} [3] {\mathop{\mathlarger{\int\limits_{#1}^{#2}}} {#3}}  % a big integral

%%%%%%%%%%%%%%%%%%%%%%%%%%%%%%%%%%%%%%%%%%%%%%%%%%%%%%%%%%%%%
%% DOCUMENT
%%%%%%%%%%%%%%%%%%%%%%%%%%%%%%%%%%%%%%%%%%%%%%%%%%%%%%%%%%%%%
\begin{document}

\pagestyle{empty} %No headings for the first pages.

%% Title Page %%%%%%%%%%%%%%%%%%%%%%%%%%%%%%%%%%%%%%%%%%%%%%%
\title{Tele-Experiment}
\subject{Kinematics of Mobile Robots}

\author{Kaung Sett Thu}
\matrnr{}
\submission{\today}

\maketitle

%% Inhaltsverzeichnis %%%%%%%%%%%%%%%%%%%%%%%%%%%%%%%%%%%%%%%
\tableofcontents
\cleardoublepage %The first chapter should start on an odd page.

\pagestyle{plain}

%% Chapters %%%%%%%%%%%%%%%%%%%%%%%%%%%%%%%%%%%%%%%%%%%%%%%%%
%\input{Chapter}
%%%%%%%%%%%%%%%%%%%%%%%%%%%%%%%%%%%%%%%%%%%%%%%%%%%%%%%%%%%%%
%% Task 3.1
%%%%%%%%%%%%%%%%%%%%%%%%%%%%%%%%%%%%%%%%%%%%%%%%%%%%%%%%%%%%%
\chapter{Simplified Car-Like Robot Model}
%
Proving the relationship between the steering angle of the virtual wheel $\varphi$ and the steering angles of the two front wheels $\varphi_{l}$ and $\varphi_{r}$.
%
\begin{figure}[H]
    \centering
    \includegraphics[width=0.6\textwidth]{../diagrams/ackarmann_robot.png}
    \caption{Simplified Car-Like Robot Model.}
    \label{fig}
\end{figure}
%
Since the track width $d$, the wheel base $l$ and the radius $r$ from the Instaneous Ceter of Curvature $ICC$ forms a right angle triangle, the steering angles can be written as:
%
\begin{align*}
    \tan \varphi &= \dfrac{l}{r} \\[1em]
    \tan \varphi_{r} &= \dfrac{l}{(r + \dfrac{d}{2} )} \\[1em]
    \tan \varphi_{l} &= \dfrac{l}{(r - \dfrac{d}{2} )}
\end{align*}
%
The equation for right steering angle can be rearranged by isolating for radius $r$.
%
\begin{align*}
    \tan \varphi_{r} &= \dfrac{l}{(r + \dfrac{d}{2} )} \\[1em]
    r + \dfrac{d}{2} &= \dfrac{l}{\tan \varphi_{r}} \\[1em]
    r &= \dfrac{l}{\tan \varphi_{r}} - \dfrac{d}{2}
\end{align*}
%
Substituting the radius into equation for steering angle of the virtual wheel,
%
\begin{align*}
    \tan \varphi &= \dfrac{l}{\dfrac{l}{\tan \varphi_{r}} - \dfrac{d}{2}}
\end{align*}
%
Dividing numerator and denominator of right hand side by $l$,
%
\begin{align*}
    \tan \varphi &= \dfrac{1}{\dfrac{1}{\tan \varphi_{r}} - \dfrac{d}{2l}} \\[1em]
    \tan \varphi &= \dfrac{1}{\left( \dfrac{2l - d \cdot \tan \varphi_{r}}{2l \cdot \tan \varphi_{r}}\right)} \\[1em]
    \tan \varphi &= \dfrac{2l \cdot \tan \varphi_{r}}{2l - d \cdot \tan \varphi_{r}}
\end{align*}
%
Dividing numerator and denominator of right hand side by $2l$,
%
\begin{align*}
    \tan \varphi &= \dfrac{\tan \varphi_{r}}{1 - \dfrac{d}{2l} \cdot \tan \varphi_{r}} \\[1em]
    \therefore \varphi &= \arctan \left( \dfrac{\tan \varphi_{r}}{1 - \dfrac{d}{2l} \cdot \tan \varphi_{r}} \right)
\end{align*}
%
%
The equation for left steering angle can be rearranged by isolating for radius $r$.
%
\begin{align*}
    \tan \varphi_{l} &= \dfrac{l}{(r - \dfrac{d}{2} )} \\[1em]
    r - \dfrac{d}{2} &= \dfrac{l}{\tan \varphi_{l}} \\[1em]
    r &= \dfrac{l}{\tan \varphi_{l}} + \dfrac{d}{2}
\end{align*}
%
Substituting the radius into equation for steering angle of the virtual wheel,
%
\begin{align*}
    \tan \varphi &= \dfrac{l}{\dfrac{l}{\tan \varphi_{l}} + \dfrac{d}{2}}
\end{align*}
%
Dividing numerator and denominator of right hand side by $l$,
%
\begin{align*}
    \tan \varphi &= \dfrac{1}{\dfrac{1}{\tan \varphi_{l}} + \dfrac{d}{2l}} \\[1em]
    \tan \varphi &= \dfrac{1}{\left( \dfrac{2l + d \cdot \tan \varphi_{l}}{2l \cdot \tan \varphi_{l}}\right)} \\[1em]
    \tan \varphi &= \dfrac{2l \cdot \tan \varphi_{l}}{2l + d \cdot \tan \varphi_{l}}
\end{align*}
%
Dividing numerator and denominator of right hand side by $2l$,
%
\begin{align*}
    \tan \varphi &= \dfrac{\tan \varphi_{l}}{1 + \dfrac{d}{2l} \cdot \tan \varphi_{l}} \\[1em]
    \therefore \varphi &= \arctan \left( \dfrac{\tan \varphi_{l}}{1 + \dfrac{d}{2l} \cdot \tan \varphi_{l}} \right)
\end{align*}
%
%
%%%%%%%%%%%%%%%%%%%%%%%%%%%%%%%%%%%%%%%%%%%%%%%%%%%%%%%%%%%%%
%% Task 3.2
%%%%%%%%%%%%%%%%%%%%%%%%%%%%%%%%%%%%%%%%%%%%%%%%%%%%%%%%%%%%%
\chapter{Pose Determination from Kinematics Model}
\section{Differential Drive Robot}
\subsection*{Case 1: Non-Straight-Line Motion}
%
Since linear velocity $v_l$ and $v_r$ are constant, the angular velocity $\omega$ is also assumed to be constnat. Therefore, instead of notating them as a function of time, they will be notated as constants $v$ and $\omega$.

\emph{Finding $\Delta \theta$ with respect to time $t_1$ and $t_2$.}
%
\begin{align*}
    \Delta \theta &= \int_{t_1}^{t_2} \omega \, dt \\[1em]
    &= \omega [t]_{t_1}^{t_2} \\[1em]
    &= \omega (t_2 - t_1)
\end{align*}
%
%% Calculation for \delta_x
%
\emph{Finding $\Delta x$ with respect to time $t_1$ and $t_2$.}
%
\begin{align*}
    \Delta x &= \int_{t_1}^{t_2} v \cdot \cos \theta(t) \, dt \\[1em]
    &= v \int_{t_1}^{t_2} \cos(\theta(t_1) + \omega(t - t_1))\, dt
\end{align*}
%
Let $u = \theta(t_1) + \omega(t - t_1)$.
%
\begin{align*}
    du &= \omega \, dt \\[1em]
    dt &= \dfrac{du}{\omega}
\end{align*}
%
Using $\theta(t_1)$ and $\theta(t_2)$ as lower and upper bounds and substituting $u$ and $du$, 
\begin{align*}
    \Delta x &= v \int_{\theta(t_1)}^{\theta(t_2)} \cos (u) \, \dfrac{du}{\omega} \\[1em]
    &= \dfrac{v}{\omega} \int_{\theta(t_1)}^{\theta(t_2)} \cos (u) \, du \\[1em]
    &= \dfrac{v}{\omega} [\sin(u)]_{\theta(t_1)}^{\theta(t_2)} \\[1em]
    &= \dfrac{v}{\omega} [\sin(\theta(t_2)) - \sin(\theta(t_1))]
\end{align*}
%
%% Calculation for \delta_y
%
\emph{Finding $\Delta y$ with respect to time $t_1$ and $t_2$.}.
%
\begin{align*}
    \Delta y &= \int_{t_1}^{t_2} v \cdot \sin \theta(t) \, dt \\[1em]
    &= v \int_{t_1}^{t_2} \sin(\theta(t_1) + \omega(t - t_1))\, dt
\end{align*}
%
Let $u = \theta(t_1) + \omega(t - t_1)$.
%
\begin{align*}
    du &= \omega \, dt \\[1em]
    dt &= \dfrac{du}{\omega}
\end{align*}
%
Using $\theta(t_1)$ and $\theta(t_2)$ as lower and upper bounds and substituting $u$ and $du$, 
\begin{align*}
    \Delta y &= v \int_{\theta(t_1)}^{\theta(t_2)} \sin (u) \, \dfrac{du}{\omega} \\[1em]
    &= \dfrac{v}{\omega} \int_{\theta(t_1)}^{\theta(t_2)} \sin (u) \, du \\[1em]
    &= \dfrac{v}{\omega} [-\cos(u)]_{\theta(t_1)}^{\theta(t_2)} \\[1em]
    &= \dfrac{v}{\omega} [\cos(\theta(t_1)) - \cos(\theta(t_2))]
\end{align*}
%
Therefore, for a differential-drive robot that is not moving in a straight line motion, 
%
\begin{align*}
    \Delta \theta &= \omega(t_2 - t_1) \\[1em]
    \Delta x &= \dfrac{v}{\omega} [\sin(\theta(t_2)) - \sin(\theta(t_1))] \\[1em]
    \Delta y &= \dfrac{v}{\omega} [\cos(\theta(t_1)) - \cos(\theta(t_2))]
\end{align*}
%
\subsection*{Case 2. Straight Line Motion}
In case of a straight line motion, the equations would not be the same. For the robot to move in a straight line motion, linear velocity $v_l$ and $v_r$ must be equal and angular velocity $\omega$ must be 0.
%

\emph{Finding $\Delta \theta$ with respect to time $t_1$ and $t_2$.}
%
\begin{align*}
    \Delta \theta &= \int_{t_1}^{t_2} \omega \, dt \\[1em]
    &= 0 [t]_{t_1}^{t_2} \\[1em]
    &= 0
\end{align*}
%
Since $\Delta \theta$ is 0, $\theta(t_1)$ remains constant.

\emph{Finding $\Delta x$ with respect to time $t_1$ and $t_2$.}
%
\begin{align*}
    \Delta x &= \int_{t_1}^{t_2} v \cdot \cos \theta(t_1) \, dt \\[1em]
    &= v \cdot \cos \theta(t_1) \int_{t_1}^{t_2} \, dt \\[1em]
    &= v \cdot \cos \theta(t_1) [t]_{t_1}^{t_2} \\[1em]
    &= v \cdot \cos \theta(t_1) \cdot (t_2 - t_1)
\end{align*}
%
\emph{Finding $\Delta y$ with respect to time $t_1$ and $t_2$.}
%
\begin{align*}
    \Delta y &= \int_{t_1}^{t_2} v \cdot \sin \theta(t_1) \, dt \\[1em]
    &= v \cdot \sin \theta(t_1) \int_{t_1}^{t_2} \, dt \\[1em]
    &= v \cdot \sin \theta(t_1) [t]_{t_1}^{t_2} \\[1em]
    &= v \cdot \sin \theta(t_1) \cdot (t_2 - t_1)
\end{align*}
%
Therefore, for a differential-drive robot that is moving in a straight line motion, 
%
\begin{align*}
    \Delta \theta &= 0 \\[1em]
    \Delta x &= v \cdot \cos \theta(t_1) \cdot (t_2 - t_1) \\[1em]
    \Delta y &= v \cdot \sin \theta(t_1) \cdot (t_2 - t_1)
\end{align*}
%
%
\section{Car-Like Robot}
\subsection*{Case 1: Non-Straight-Line Motion}
Steering angle $\varphi$ and linear velocity $v$ are constant.

\emph{Finding $\Delta \theta$ with respect to time $t_1$ and $t_2$.}
%
\begin{align*}
    \Delta \theta &= \int_{t_1}^{t_2} v \cdot \dfrac{\tan \varphi}{l} \, dt \\[1em]
    &= v \cdot \dfrac{\tan \varphi}{l} \int_{t_1}^{t_2} \, dt \\[1em]
    &= v \cdot \dfrac{\tan \varphi}{l} [t]_{t_1}^{t_2} \\[1em]
    &= v \cdot \dfrac{\tan \varphi}{l} \, (t_2 - t_1)
\end{align*}
%
%% Calculating \delta_x
%
\emph{Finding $\Delta x$ with respect to time $t_1$ and $t_2$.}
%
\begin{align*}
    \Delta x &= \int_{t_1}^{t_2} v \cdot \cos \theta(t) \, dt \\[1em]
    &= v \int_{t_1}^{t_2} \cos \left(\theta(t_1) + v \cdot \dfrac{\tan \varphi}{l} \, (t - t_1)\right) \, dt
\end{align*}
%
Let $u = \theta(t_1) + v \cdot \dfrac{\tan \varphi}{l} \, (t - t_1)$.
%
\begin{align*}
    du &= \dfrac{v \cdot \tan \varphi}{l} \, dt \\[1em]
    dt &= \dfrac{l}{v \cdot \tan \varphi} \, du
\end{align*}
%
Using $\theta(t_1)$ and $\theta(t_2)$ as lower and upper bounds and substituting $u$ and $du$, 
\begin{align*}
    \Delta x &= v \int_{\theta(t_1)}^{\theta(t_2)} \cos (u) \cdot \dfrac{l}{v \cdot \tan \varphi} \, du \\[1em]
    &= \dfrac{l}{\tan \varphi} \int_{\theta(t_1)}^{\theta(t_2)} \cos (u) \, du \\[1em]
    &= \dfrac{l}{\tan \varphi} [\sin(u)]_{\theta(t_1)}^{\theta(t_2)} \\[1em]
    &= \dfrac{l}{\tan \varphi} [\sin(\theta(t_2)) - \sin(\theta(t_1))]
\end{align*}
%
%
%% Calculating \delta_y
%
\emph{Finding $\Delta y$ with respect to time $t_1$ and $t_2$.}
%
\begin{align*}
    \Delta y &= \int_{t_1}^{t_2} v \cdot \sin \theta(t) \, dt \\[1em]
    &= v \int_{t_1}^{t_2} \sin \left(\theta(t_1) + v \cdot \dfrac{\tan \varphi}{l} \, (t - t_1)\right) \, dt
\end{align*}
%
Let $u = \theta(t_1) + v \cdot \dfrac{\tan \varphi}{l} \, (t - t_1)$.
%
\begin{align*}
    du &= \dfrac{v \cdot \tan \varphi}{l} \, dt \\[1em]
    dt &= \dfrac{l}{v \cdot \tan \varphi} \, du
\end{align*}
%
Using $\theta(t_1)$ and $\theta(t_2)$ as lower and upper bounds and substituting $u$ and $du$, 
\begin{align*}
    \Delta x &= v \int_{\theta(t_1)}^{\theta(t_2)} \sin (u) \cdot \dfrac{l}{v \cdot \tan \varphi} \, du \\[1em]
    &= \dfrac{l}{\tan \varphi} \int_{\theta(t_1)}^{\theta(t_2)} \sin (u) \, du \\[1em]
    &= \dfrac{l}{\tan \varphi} [-\cos(u)]_{\theta(t_1)}^{\theta(t_2)} \\[1em]
    &= \dfrac{l}{\tan \varphi} [\cos(\theta(t_1)) - \cos(\theta(t_2))]
\end{align*}
%
Therefore, for a differential-drive robot that is moving in a straight line motion, 
%
\begin{align*}
    \Delta \theta &= v \cdot \dfrac{\tan \varphi}{l} \, (t_2 - t_1) \\[1em]
    \Delta x &= \dfrac{l}{\tan \varphi} [\sin(\theta(t_2)) - \sin(\theta(t_1))] \\[1em]
    \Delta y &= \dfrac{l}{\tan \varphi} [\cos(\theta(t_1)) - \cos(\theta(t_2))]
\end{align*}
%
%
\subsection*{Case 2. Straight Line Motion}
The equations for Straight line motion for the car-like robot will also be different from the non-straight-line motion. For the robot moving in straight line motion, the steering angle $\varphi$ is 0.

\emph{Finding $\Delta \theta$ with respect to time $t_1$ and $t_2$.}
%
\begin{align*}
    \Delta \theta &= \int_{t_1}^{t_2} v \cdot \dfrac{\tan \varphi}{l} \, dt \\[1em]
    &= v \cdot \dfrac{\tan \varphi}{l} \int_{t_1}^{t_2} \, dt \\[1em]
    &= 0 [t]_{t_1}^{t_2} \\[1em]
    &= 0
\end{align*}
%
Since $\Delta \theta$ is 0, $\theta(t_1)$ remains constant.

\emph{Finding $\Delta x$ with respect to time $t_1$ and $t_2$.}
%
\begin{align*}
    \Delta x &= \int_{t_1}^{t_2} v \cdot \cos \theta(t_1) \, dt \\[1em]
    &= v \cdot \cos \theta(t_1) \int_{t_1}^{t_2} \, dt \\[1em]
    &= v \cdot \cos \theta(t_1) [t]_{t_1}^{t_2} \\[1em]
    &= v \cdot \cos \theta(t_1) \cdot (t_2 - t_1)
\end{align*}
%
\emph{Finding $\Delta y$ with respect to time $t_1$ and $t_2$.}
%
\begin{align*}
    \Delta y &= \int_{t_1}^{t_2} v \cdot \sin \theta(t_1) \, dt \\[1em]
    &= v \cdot \sin \theta(t_1) \int_{t_1}^{t_2} \, dt \\[1em]
    &= v \cdot \sin \theta(t_1) [t]_{t_1}^{t_2} \\[1em]
    &= v \cdot \sin \theta(t_1) \cdot (t_2 - t_1)
\end{align*}
%
Therefore, for a car-like robot that is moving in a straight line motion, 
%
\begin{align*}
    \Delta \theta &= 0 \\[1em]
    \Delta x &= v \cdot \cos \theta(t_1) \cdot (t_2 - t_1) \\[1em]
    \Delta y &= v \cdot \sin \theta(t_1) \cdot (t_2 - t_1)
\end{align*}
%
%%%%%%%%%%%%%%%%%%%%%%%%%%%%%%%%%%%%%%%%%%%%%%%%%%%%%%%%%%%%%
%% Task 3.3
%%%%%%%%%%%%%%%%%%%%%%%%%%%%%%%%%%%%%%%%%%%%%%%%%%%%%%%%%%%%%
\chapter{Kinematics of Bicycles and Tricycles}

The kinematic differences between the Ackermann steering robot and the standard bicycle or tricycle model can be seen the in the following table.
%
\begin{table}[htbp]
\centering
\renewcommand{\arraystretch}{1.3}
\begin{tabularx}{\textwidth}{X X}
    \toprule
    \textbf{Ackermann Steering} & \textbf{Bicycles and Tricycles} \\
    \midrule
    two forward steering wheels & only one single steering front wheel \\
    two independent steering angles $\varphi_{l}$ and $\varphi_{r}$ and a virtual steering angle $\varphi$ & only one steering angle $\varphi$ \\
    the orientation of both front wheels must be perpendicular to ICC & the orientation of front wheel must be perpendicular to ICC $\varphi$ \\
    \bottomrule
\end{tabularx}
\caption{Differences between Ackermann steering and standard bicycles or tricycles}
\end{table}

When the ICC falls between the rear wheels in a rear wheel driven tricycle, assuming that the steering angle is not zero ($\varphi \neq 0$) the robot will move in circles around the ICC.
%
%%%%%%%%%%%%%%%%%%%%%%%%%%%%%%%%%%%%%%%%%%%%%%%%%%%%%%%%%%%%%
%% Task 3.4
%%%%%%%%%%%%%%%%%%%%%%%%%%%%%%%%%%%%%%%%%%%%%%%%%%%%%%%%%%%%%
\chapter{Inverse Kinematics}
%
The angular velocity $\omega$ is the change in the orientation $\Delta \theta$ over the period between times $t_1$ and $t_2$. Therefore it can be described as
\begin{align*}
    \omega = \dfrac{\Delta \theta}{\Delta t}
\end{align*}
%
The linear velocity is the distance the robot has travelled over the arc length $s$ over the period between times $t_1$ and $t_2$ and therefore can be described as
\begin{align*}
    v = \dfrac{s}{\Delta t}
\end{align*}
The arc length can be calculate by the use of the radius and the angle difference using the forumla $s = r \cdot \theta$. Therefore, the linear velocity $v$ can be rewritten as
%
\begin{align*}
    v &= \dfrac{r \cdot \Delta \theta}{\Delta t} \\[1em]
    &= r \cdot \omega
\end{align*}
%
%
%
The angular velocity $w$ and the linear velocity $v$ of a differential drive robot can also be calculated from the linear velocity of the left and the right wheels of the robot, $v_l$ and $v_r$ using the formulas
\begin{align*}
    \omega &= \dfrac{1}{d} (v_r - v_l) \\[1em]
    v &= \dfrac{1}{2} (v_r + v_l)
\end{align*}
where $d$ is the track width of the robot.

The equation for linear velocity $v$ can be rearranged to isolate $v_l$.
%
\begin{align*}
    2v &= v_r + v_l \\[1em]
    v_l &= 2v - v_r
\end{align*}
%
Substituting the equation into the equation for angular velocity $w$.
%
\begin{align*}
    \omega &= \dfrac{1}{d} (v_r - (2v - v_r)) \\[1em]
    d \cdot \omega &= v_r - (2v - v_r) \\[1em]
    d \cdot \omega &= 2v_r - 2v \\[1em]
    2v_r &= 2v +  d \cdot \omega \\[1em]
    v_r &= v + \dfrac{d \cdot \omega}{2}
\end{align*}
%
Substituting it back to the equation for $v_l$.
%
\begin{align*}
    v_l &= 2v - \left(v + \dfrac{d \cdot \omega}{2}\right) \\[1em]
    v_l &= v - \dfrac{d \cdot \omega}{2}
\end{align*}
%
%
\section{Car-Like Robot}
%
The linear velocity $v$ of the imaginary wheel of the car-like robot is the distance it travelled over the period between times $t_1$ and $t_2$ and therefore can also be described as
\begin{align*}
    v = r \cdot \omega
\end{align*}
%
The relationship between the virtual steering angle $\varphi$, wheel base $l$ and track width $d$ can be described as
%
\begin{align*}
    \tan \varphi = \dfrac{l}{r}
\end{align*}
%
Therefore the value of the virtual steering angle can be calculated by simply rearranging the formula.
%
\begin{align*}
    \varphi = \arctan \left(\dfrac{l}{r}\right)
\end{align*}
%
%
%%%%%%%%%%%%%%%%%%%%%%%%%%%%%%%%%%%%%%%%%%%%%%%%%%%%%%%%%%%%%
%% Task 3.5
%%%%%%%%%%%%%%%%%%%%%%%%%%%%%%%%%%%%%%%%%%%%%%%%%%%%%%%%%%%%%
\chapter{Simulation of Kinematics}
%
\section{Ackermann steering robot}
Before determing a set of maneuvers for the Ackermann steered robot, the minimum radius that is physically possible from the ICC needs to be calculated to ensure that the maneuver follows the kinematic constants. Since the maximum steering angle is $25^\circ$, the minimum radius can be calculated assumed
%
\begin{align*}
    r_{min} &= \dfrac{l}{\tan \varphi} \\[1em]
    &= \dfrac{2}{\tan 25^\circ} \\[1em]
    &= 4.29
\end{align*}

\subsection{Scence 1}
%
Initial pose: $p_i = \begin{bmatrix} 5 & 5 & \dfrac{\pi}{2} \end{bmatrix}^{\mathsf T}$ 

Final pose: $p_g = \begin{bmatrix} 17 & 5 & -\dfrac{\pi}{2} \end{bmatrix}^{\mathsf T}$
%
\subsubsection{Set of maneuvers}
\begin{enumerate}
    \item Turn right along a circular arc of radius 6 sq through $\pi$ rad for 2 s.
\end{enumerate}
%
\subsubsection{Commands}
\emph{For maneuver 1,}
\begin{align*}
    \omega &= \dfrac{\Delta \theta}{\Delta t} = \dfrac{\pi}{2} \\[1em]
    v &= r \cdot \omega = 6 \cdot \dfrac{\pi}{2} = 9.4248\, sq/s \\[1em]
    \varphi &= \arctan \left(\dfrac{l}{r}\right) = \arctan \left(\dfrac{2}{6}\right) = 18.43^\circ
\end{align*}
%
Since the robot is turning right, the steering angle $\varphi$ is $-18.43^\circ$.
%
\subsubsection{Simulation}
\begin{figure}[H]
    \centering
    \includegraphics[width=1.0\textwidth]{../screenshots/ackerman-scene1.png}
\end{figure}
%
%
\subsection{Scence 2}
%
Initial pose: $p_i = \begin{bmatrix} 5 & 5 & \dfrac{\pi}{2} \end{bmatrix}^{\mathsf T}$ 

Final pose: $p_g = \begin{bmatrix} 17 & 3 & \dfrac{\pi}{2} \end{bmatrix}^{\mathsf T}$
%
\subsubsection{Set of maneuvers}
\begin{enumerate}
    \item Turn right along a circular arc of radius 5 sq through $\dfrac{\pi}{2}$ rad for 1 s.
    \item Travel in a straight line of 10 sq for 2 s.
    \item Turn right along a circular arc of radius 7 sq through $\dfrac{\pi}{2}$ rad for 2 s.
    \item Turn left backwards along a circular arc of radius 5 sq through $\pi$ rad for 2 s.
\end{enumerate}
%
\subsubsection{Commands}
\emph{For maneuver 1,}
\begin{align*}
    \omega &= \dfrac{\Delta \theta}{\Delta t} = \dfrac{\pi / 2}{1}  = \dfrac{\pi}{2}\, rad/s\\[1em]
    v &= r \cdot \omega = 5 \cdot \dfrac{\pi}{2} = 7.854\, sq/s \\[1em]
    \varphi &= \arctan \left(\dfrac{l}{r}\right) = \arctan \left(\dfrac{2}{5}\right) = 21.8^\circ
\end{align*}
%
Since the robot is turning right, the steering angle $\varphi$ is $-21.8^\circ$.

\emph{For maneuver 2,}
\begin{align*}
    v &= \dfrac{s}{\Delta t} = \dfrac{10}{2} = 5\, sq/s \\[1em]
    \varphi &= 0^\circ
\end{align*}

\emph{For maneuver 3,}
\begin{align*}
    \omega &= \dfrac{\Delta \theta}{\Delta t} = \dfrac{\pi / 2}{2}  = \dfrac{\pi}{4}\, rad/s\\[1em]
    v &= r \cdot \omega = 7 \cdot \dfrac{\pi}{4} = 5.4978\, sq/s \\[1em]
    \varphi &= \arctan \left(\dfrac{l}{r}\right) = \arctan \left(\dfrac{2}{7}\right) = 15.95^\circ
\end{align*}
%
Since the robot is turning right, the steering angle $\varphi$ is $-15.95^\circ$.

\emph{For maneuver 4,}
\begin{align*}
    \omega &= \dfrac{\Delta \theta}{\Delta t} = \dfrac{\pi}{2}\, rad/s\\[1em]
    v &= r \cdot \omega = 5 \cdot \dfrac{\pi}{2} = 7.854\, sq/s \\[1em]
    \varphi &= \arctan \left(\dfrac{l}{r}\right) = \arctan \left(\dfrac{2}{5}\right) = 21.8^\circ
\end{align*}
%
Since the robot is turning left backwards, the linear velocity $v$ is -7.854 sq/s and the steering angle $\varphi$ is $-21.8^\circ$.

\subsubsection{Simulation}
\begin{figure}[H]
    \centering
    \includegraphics[width=1.0\textwidth]{../screenshots/ackerman-scene2.png}
\end{figure}
%
%
\subsection{Scence 3}
%
Initial pose: $p_i = \begin{bmatrix} 5 & 5 & \dfrac{\pi}{4} \end{bmatrix}^{\mathsf T}$ 

Final pose: $p_g = \begin{bmatrix} 17 & 3 & \dfrac{\pi}{2} \end{bmatrix}^{\mathsf T}$
%
\subsubsection{Set of maneuvers}
\begin{enumerate}
    \item Turn right along a circular arc of radius 5 sq through $\dfrac{\pi}{4}$ rad for 1 s.
    \item Travel in a straight line of 5.5 sq for 1 s.
    \item Turn left along a circular arc of radius 5 sq through $\dfrac{\pi}{2}$ rad for 1 s.
    \item Travel in a straight line of 4.5 sq for 1 s.
\end{enumerate}
%
\subsubsection{Commands}
\emph{For maneuver 1,}
\begin{align*}
    \omega &= \dfrac{\Delta \theta}{\Delta t} = \dfrac{\pi / 4}{1}  = \dfrac{\pi}{4}\, rad/s\\[1em]
    v &= r \cdot \omega = 5 \cdot \dfrac{\pi}{4} = 3.927\, sq/s \\[1em]
    \varphi &= \arctan \left(\dfrac{l}{r}\right) = \arctan \left(\dfrac{2}{5}\right) = 21.8^\circ
\end{align*}
%
Since the robot is turning right, the steering angle $\varphi$ is $-21.8^\circ$.

\emph{For maneuver 2,}
\begin{align*}
    v &= \dfrac{s}{\Delta t} = \dfrac{5.5}{1} = 5.5\, sq/s \\[1em]
    \varphi &= 0^\circ
\end{align*}

\emph{For maneuver 3,}
\begin{align*}
    \omega &= \dfrac{\Delta \theta}{\Delta t} = \dfrac{\pi/2}{1} = \dfrac{\pi}{2}\, rad/s\\[1em]
    v &= r \cdot \omega = 5 \cdot \dfrac{\pi}{2} = 7.854\, sq/s \\[1em]
    \varphi &= \arctan \left(\dfrac{l}{r}\right) = \arctan \left(\dfrac{2}{5}\right) = 21.8^\circ
\end{align*}

\emph{For maneuver 4,}
\begin{align*}
    v &= \dfrac{s}{\Delta t} = \dfrac{5.5}{1} = 4.5\, sq/s \\[1em]
    \varphi &= 0^\circ
\end{align*}

\subsubsection{Simulation}
\begin{figure}[H]
    \centering
    \includegraphics[width=1.0\textwidth]{../screenshots/ackerman-scene3.png}
\end{figure}
%
%
\subsection{Scence 4}
%
Initial pose: $p_i = \begin{bmatrix} 7 & 14 & \dfrac{\pi}{2} \end{bmatrix}^{\mathsf T}$ 

Final pose: $p_g = \begin{bmatrix} 11 & 8 & \dfrac{\pi}{2} \end{bmatrix}^{\mathsf T}$
%
\subsubsection{Set of maneuvers}
\begin{enumerate}
    \item Travel backwards in a straight line of 7.5 sq for 1 s.
    \item Turn left along a circular arc of radius 4.5 sq through $\dfrac{\pi}{2}$ rad for 1 s.
    \item Travel backwards in a straight line of 4 sq for 0.5 s.
    \item Turn right backwards along a circular arc of radius 4.5 sq through $\dfrac{\pi}{2}$ rad for 1 s.
    \item Travel in a straight line of 1.5 sq for 0.5 s.
\end{enumerate}
%
\subsubsection{Commands}

\emph{For maneuver 1,}
\begin{align*}
    v &= \dfrac{s}{\Delta t} = \dfrac{7.5}{1} = 7.5\, sq/s \\[1em]
    \varphi &= 0^\circ
\end{align*}
%
Since the robot is moving backwards, the linear velocity $v$ is -7.5 sq/s.

\emph{For maneuver 2,}
\begin{align*}
    \omega &= \dfrac{\Delta \theta}{\Delta t} = \dfrac{\pi / 2}{1}  = \dfrac{\pi}{2}\, rad/s\\[1em]
    v &= r \cdot \omega = 4.5 \cdot \dfrac{\pi}{4} = 7.0686\, sq/s \\[1em]
    \varphi &= \arctan \left(\dfrac{l}{r}\right) = \arctan \left(\dfrac{2}{4.5}\right) = 23.96^\circ
\end{align*}

\emph{For maneuver 3,}
\begin{align*}
    v &= \dfrac{s}{\Delta t} = \dfrac{4}{0.5} = 8.0\, sq/s \\[1em]
    \varphi &= 0^\circ
\end{align*}
%
Since the robot is moving backwards, the linear velocity $v$ is -8.0 sq/s.

\emph{For maneuver 4,}
\begin{align*}
    \omega &= \dfrac{\Delta \theta}{\Delta t} = \dfrac{\pi / 2}{1}  = \dfrac{\pi}{2}\, rad/s\\[1em]
    v &= r \cdot \omega = 4.5 \cdot \dfrac{\pi}{4} = 7.0686\, sq/s \\[1em]
    \varphi &= \arctan \left(\dfrac{l}{r}\right) = \arctan \left(\dfrac{2}{4.5}\right) = 23.96^\circ
\end{align*}
Since the robot is moving backwards, the linear velocity $v$ is -7.0686 sq/s.

\emph{For maneuver 5,}
\begin{align*}
    v &= \dfrac{s}{\Delta t} = \dfrac{1.5}{0.5} = 3.0\, sq/s \\[1em]
    \varphi &= 0^\circ
\end{align*}
%
\subsubsection{Simulation}
\begin{figure}[H]
    \centering
    \includegraphics[width=1.0\textwidth]{../screenshots/ackerman-scene4.png}
\end{figure}
%

\section{Differential-drive robot}
\subsection{Scence 3}
%
Initial pose: $p_i = \begin{bmatrix} 5 & 5 & \dfrac{\pi}{4} \end{bmatrix}^{\mathsf T}$ 

Final pose: $p_g = \begin{bmatrix} 17 & 3 & \dfrac{\pi}{2} \end{bmatrix}^{\mathsf T}$
%
\subsubsection{Set of maneuvers}
\begin{enumerate}
    \item Rotate right through $\dfrac{\pi}{4}$ rad for 0.5 s.
    \item Travel in a straight line of 14 sq for 2 s.
    \item Rotate left through $\dfrac{\pi}{2}$ rad for 1 s.
    \item Travel in a straight line of 11 sq for 2 s.
\end{enumerate}
%
\subsubsection{Commands}

\emph{For maneuver 1,}
\begin{align*}
    \omega &= \dfrac{\Delta \theta}{\Delta t} = \dfrac{- \pi / 4}{0.5}  = - \dfrac{\pi}{2}\, rad/s \\[1em]
    v &= 0\, sq/s\\[1em]
    v_r &= v + \dfrac{d \cdot \omega}{2} = 0 + \dfrac{1 \cdot ( - \pi / 2)}{2} = - 1.5708\, sq/s \\[1em]
    v_l &= v - \dfrac{d \cdot \omega}{2} = 0 - \dfrac{1 \cdot ( - \pi / 2)}{2} = 1.5708\, sq/s
\end{align*}

\emph{For maneuver 2,}
\begin{align*}
    \omega &= 0\, rad/s \\[1em]
    v &= \dfrac{s}{\Delta t} = \dfrac{14}{2} = 7\, sq/s \\[1em]
    v_r &= v + \dfrac{d \cdot \omega}{2} = 7 + \dfrac{1 \cdot 0}{2} = 7\, sq/s \\[1em]
    v_l &= v - \dfrac{d \cdot \omega}{2} = 7 - \dfrac{1 \cdot 0}{2} = 7\, sq/s
\end{align*}

\emph{For maneuver 3,}
\begin{align*}
    \omega &= \dfrac{\Delta \theta}{\Delta t} = \dfrac{\pi / 2}{1}  = \dfrac{\pi}{2}\, rad/s \\[1em]
    v &= 0\, sq/s\\[1em]
    v_r &= v + \dfrac{d \cdot \omega}{2} = 0 + \dfrac{1 \cdot \pi / 2}{2} = 1.5708\, sq/s \\[1em]
    v_l &= v - \dfrac{d \cdot \omega}{2} = 0 - \dfrac{1 \cdot \pi / 2}{2} = -1.5708\, sq/s
\end{align*}

\emph{For maneuver 4,}
\begin{align*}
    \omega &= 0\, rad/s \\[1em]
    v &= \dfrac{s}{\Delta t} = \dfrac{11}{2} = 5.5\, sq/s \\[1em]
    v_r &= v + \dfrac{d \cdot \omega}{2} = 5.5 + \dfrac{1 \cdot 0}{2} = 5.5\, sq/s \\[1em]
    v_l &= v - \dfrac{d \cdot \omega}{2} = 5.5 - \dfrac{1 \cdot 0}{2} = 5.5\, sq/s
\end{align*}
%
\subsubsection{Simulation}
\begin{figure}[H]
    \centering
    \includegraphics[width=1.0\textwidth]{../screenshots/differential-scene3.png}
\end{figure}
%
%
\subsection{Scence 4}
%
Initial pose: $p_i = \begin{bmatrix} 7 & 14 & \dfrac{\pi}{2} \end{bmatrix}^{\mathsf T}$ 

Final pose: $p_g = \begin{bmatrix} 11 & 8 & \dfrac{\pi}{2} \end{bmatrix}^{\mathsf T}$
%
\subsubsection{Set of maneuvers}
\begin{enumerate}
    \item Travel backwards in a straight line of 2 sq for 0.5 s.
    \item Rotate left through $\dfrac{\pi}{4}$ rad for 0.5 s.
    \item Travel backwards in a straight line of $\left(\sqrt{{\Delta x}^2 + {\Delta y}^2} = \sqrt{{4}^2 + {4}^2}\right)$ 5.66 sq for 1 s.
    \item Rotate left through $\dfrac{\pi}{2}$ rad for 1 s.
\end{enumerate}
%
\subsubsection{Commands}

\emph{For maneuver 1,}
\begin{align*}
    \omega &= 0\, rad/s \\[1em]
    v &= \dfrac{s}{\Delta t} = \dfrac{-2}{0.5} = -4\, sq/s \\[1em]
    v_r &= v + \dfrac{d \cdot \omega}{2} = -4 + \dfrac{1 \cdot 0}{2} = -4\, sq/s \\[1em]
    v_l &= v - \dfrac{d \cdot \omega}{2} = -4 - \dfrac{1 \cdot 0}{2} = -4\, sq/s
\end{align*}

\emph{For maneuver 2,}
\begin{align*}
    \omega &= \dfrac{\Delta \theta}{\Delta t} = \dfrac{\pi / 4}{0.5}  = \dfrac{\pi}{2}\, rad/s \\[1em]
    v &= 0\, sq/s\\[1em]
    v_r &= v + \dfrac{d \cdot \omega}{2} = 0 + \dfrac{1 \cdot \pi / 2}{2} = 1.5708\, sq/s \\[1em]
    v_l &= v - \dfrac{d \cdot \omega}{2} = 0 - \dfrac{1 \cdot \pi / 2}{2} = -1.5708\, sq/s
\end{align*}

\emph{For maneuver 3,}
\begin{align*}
    \omega &= 0\, rad/s \\[1em]
    v &= \dfrac{s}{\Delta t} = \dfrac{-6.4}{1} = -6.4\, sq/s \\[1em]
    v_r &= v + \dfrac{d \cdot \omega}{2} = 8 + \dfrac{1 \cdot 0}{2} = 8\, sq/s \\[1em]
    v_l &= v - \dfrac{d \cdot \omega}{2} = 8 - \dfrac{1 \cdot 0}{2} = 8\, sq/s
\end{align*}

\emph{For maneuver 4,}
\begin{align*}
    \omega &= \dfrac{\Delta \theta}{\Delta t} = \dfrac{- \pi / 4}{0.5}  = -\dfrac{\pi}{2}\, rad/s \\[1em]
    v &= 0\, sq/s\\[1em]
    v_r &= v + \dfrac{d \cdot \omega}{2} = 0 + \dfrac{1 \cdot (- \pi / 2)}{2} = -1.5708\, sq/s \\[1em]
    v_l &= v - \dfrac{d \cdot \omega}{2} = 0 - \dfrac{1 \cdot (- \pi / 2)}{2} = 1.5708\, sq/s
\end{align*}

%
\subsubsection{Simulation}
\begin{figure}[H]
    \centering
    \includegraphics[width=1.0\textwidth]{../screenshots/differential-scene4.png}
\end{figure}


%%%%%%%%%%%%%%%%%%%%%%%%%%%%%%%%%%%%%%%%%%%%%%%%%%%%%%%%%%%%%
%% BIBLIOGRAPHY AND OTHER LISTS
%%%%%%%%%%%%%%%%%%%%%%%%%%%%%%%%%%%%%%%%%%%%%%%%%%%%%%%%%%%%%
\bibliographystyle{plain}
%\bibliography{Literature}

%%%%%%%%%%%%%%%%%%%%%%%%%%%%%%%%%%%%%%%%%%%%%%%%%%%%%%%%%%%%%
%% Appendix
%%%%%%%%%%%%%%%%%%%%%%%%%%%%%%%%%%%%%%%%%%%%%%%%%%%%%%%%%%%%%
\begin{appendix}

%\input{Appendix}

\end{appendix}

\end{document}

